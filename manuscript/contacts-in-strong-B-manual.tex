\documentclass[aps,prd,onecolumn
,tightenlines,letterpaper,notitlepage,
%superscriptaddress,
nofootinbib]{revtex4-1}
\usepackage{amssymb,latexsym}
\usepackage{amsmath,amsbsy,bbm}
\usepackage{epsfig,bm,color}
\usepackage{cjhebrew}
\usepackage{nicefrac}
\usepackage{graphicx,comment}
\usepackage{slashed}
\usepackage{bbold}
\usepackage{lipsum,pdflscape}

%\usepackage{hyperref}
\unitlength=1mm
\DeclareMathOperator{\st}{str}
\DeclareMathOperator{\Erfc}{Erfc}
\DeclareMathOperator{\Erf}{Erf}
\DeclareMathOperator{\Pf}{Pf}
\DeclareMathOperator{\sign}{sign}

\usepackage{dutchcal}

\usepackage{calligra}

\DeclareMathAlphabet{\mathcalligra}{T1}{calligra}{m}{n}
\DeclareFontShape{T1}{calligra}{m}{n}{<->s*[2.2]callig15}{}
\newcommand{\scriptr}{\mathcalligra{r}\,}
\newcommand{\boldscriptr}{\pmb{\mathcalligra{r}}\,}

\begin{document}

\def\a{{\alpha}}
\def\b{{\beta}}
\def\d{{\delta}}
\def\D{{\Delta}}
\def\X{{\Xi}}
\def\e{{\varepsilon}}
\def\g{{\gamma}}
\def\G{{\Gamma}}
\def\k{{\kappa}}
\def\l{{\lambda}}
\def\L{{\Lambda}}
\def\m{{\mu}}
\def\n{{\nu}}
\def\o{{\omega}}
\def\O{{\Omega}}
\def\S{{\Sigma}}
\def\s{{\sigma}}
\def\th{{\theta}}

\def\ol#1{{\overline{#1}}}

\def\Dslash{D\hskip-0.65em /}
\def\Dtslash{\tilde{D} \hskip-0.65em /}

\def\order{{\mathcal O}}

\def\Dt{{\tilde{D}}}
\def\St{{\tilde{\Sigma}}}

\def\eqref#1{{(\ref{#1})}}

\newcommand{\caf}{\text{\cjRL{b}}}
\newcommand{\he}{${}^4$He}
\newcommand{\hes}{${}^3$He}
\newcommand{\tr}{${}^3$H}
\newcommand{\ls}{\ve{L}\cdot\ve{S}}
\newcommand{\eps}{\epsilon}
\newcommand{\as}{a_s}
\newcommand{\at}{a_t}
\newcommand{\ecm}{E_\textrm{\small c.m.}}
\newcommand{\dq}{\mbox{d\hspace{-.55em}$^-$}}
\newcommand{\mpis}{$m_\pi=137~${\small MeV}}
\newcommand{\mpim}{$m_\pi=450~${\small MeV}}
\newcommand{\mpil}{$m_\pi=806~${\small MeV}}
\newcommand{\muh}{\mu_{^3\text{\scriptsize He}}}
\newcommand{\mut}{\mu_{^3\text{\scriptsize H}}}
\newcommand{\mud}{\mu_\text{\scriptsize D}}
\newcommand{\pode}{\beta_{\text{\scriptsize D},\pm1}}
\newcommand{\poh}{\beta_{^3\text{\scriptsize He}}}
\newcommand{\pot}{\beta_{^3\text{\scriptsize H}}}
\newcommand{\com}[1]{{\scriptsize \sffamily \bfseries \color{red}{#1}}}
\newcommand{\eg}{\textit{e.g.}\;}
\newcommand{\ie}{\textit{i.e.}\;}
\newcommand{\cf}{\textit{c.f.}\;}
\newcommand{\be}{\begin{equation}}
\newcommand{\ee}{\end{equation}}
\newcommand{\la}{\label}
\newcommand{\ber}{\begin{eqnarray}}
\newcommand{\eer}{\end{eqnarray}}
\newcommand{\nn}{\nonumber}
\newcommand{\half}{\frac{1}{2}}
\newcommand{\thalf}{\nicefrac[]{3}{2}}
\newcommand{\bs}[1]{\ensuremath{\boldsymbol{#1}}}
\newcommand{\bea}{\begin{eqnarray}}
\newcommand{\eea}{\end{eqnarray}}
\newcommand{\beq}{\begin{align}}
\newcommand{\eeq}{\end{align}}
\newcommand{\bk}{\bs k}
\newcommand{\bt}{B_{^{3}\text{H}}}
\newcommand{\bh}{B_{^{3}\text{He}}}
\newcommand{\bd}{B_\text{D}}
\newcommand{\ba}{B_\alpha}
\newcommand{\rgm}{$\mathbb{R}$GM}
\newcommand{\bra}[1] {\left\langle~#1~\right|}
\newcommand{\ket}[1] {\left|~#1~\right\rangle}
\newcommand{\overlap}[2] {\left\langle~#1~\left|~#2~\right.\right\rangle}
\newcommand{\me}[3] {\left\langle~#1~\left|~#2~\left|~#3~\right.\right.\right\rangle}
\newcommand{\lam}[1]{$\Lambda=#1~$fm$^{-1}$}
\newcommand{\parg}[1] {\paragraph*{-\,\textit{#1}\,-}}
\newcommand{\nopi}{\pi\hspace{-6pt}/}
\newcommand{\ve}[1]{\ensuremath{\boldsymbol{#1}}}
\newcommand{\xvec}{\bs{x}}
\newcommand{\rvec}{\bs{r}}
\newcommand{\sgve}{\ensuremath{\boldsymbol{\sigma}}}
\newcommand{\tave}{\ensuremath{\boldsymbol{\tau}}}
\newcommand{\na}{\nabla}
\newcommand{\sumin}{\sum\limits_{i=1}^N}
\newcommand{\sumijn}{\sum\limits_{i<j}^N}
\newcommand{\sumijk}{\sum_\text{cyc.}\sum_{i<j<k}}
\newcommand{\prodcart}{\prod\limits_{c=x,y,z}}
\newcommand{\tx}{\tilde{x}}
\newcommand{\eftnopi}{\mbox{EFT($\slashed{\pi}$)}}

\newcommand{\cmment}[2]{\paragraph*{Ecce: #1}\texttt{\textcolor{blue}{#2}}}



\author{$\mathcal{M}$.~$\mathcal{Elyahu}$}
\author{$\mathcal{N}$.~$\mathcal{Barnea}$}
\author{$\mathcal{J}$.~$\mathcal{Kirscher}$}
%\email[]{$\texttt{bctiburz@chinamail.cn}$}
%\affiliation{%
%$\mathcal{CCNY}$ (Consolidated Chinese National Yeshiva)}

\title{
Strong magnetic fields and contact interactions in few-fermion systems
} 

\begin{abstract}
Technical manual detailing the implementation of a variational solution of the non-relativistic few-body
problem in an external, \ie, static magnetic field.
\end{abstract}

\pacs{}

\maketitle

\paragraph{The symmetric Gauge}
\be\la{eq.symmgauge}
\ve{A}_i=\frac{B_0}{2}(-y_i,x_i,0)
\ee

\paragraph{The Hamiltonian}
\begin{align}\la{eq.hamiltonian}
\hat{H}=-\frac{\hbar^2}{2m}&\sumin\Bigg\lbrace
\ve{\nabla}_i^2
+i\left(\frac{\hbar^2}{2m}\right)\left(\frac{q_iB_0}{\hbar}\right)
L_i^z
+\left(\frac{\hbar^2}{2m}\right)\left(\frac{q_iB_0}{\hbar}\right)^2\frac{1}{4}
\left(x_i^2+y_i^2\right)
-g_i\left(\frac{\hbar^2}{2m}\right)\left(\frac{q_iB_0}{\hbar}\right)\sigma_{z_i}
\Bigg\rbrace\\
&+~\sumijn\left[C_a +
C_b(\sigma^+_i\sigma^-_j+\sigma^-_i\sigma^+_j-\sigma^z_i\sigma^z_j)\right]~e^{-\frac{\Lambda^2}{4}(\ve{r}_i-\ve{r}_j)^2}
+\sumijk D\cdot
e^{-\frac{\Lambda^2}{4}
\left((\ve{r}_i-\ve{r}_j)^2+(\ve{r}_i-\ve{r}_k)^2\right)}
\end{align}

\paragraph{The variational basis}
\be\la{eq.svmME}
\ket{A,\ve{\lambda}}:=
e^{-\frac{1}{2}\ve{x}^TA_x\ve{x}}~
e^{-\frac{1}{2}\ve{y}^TA_y\ve{y}}~
e^{-\frac{1}{2}\ve{z}^TA_z\ve{z}}\cdot
\sum\limits_\alpha^{|\lambda|}\lambda_\alpha\underbrace{\sum\limits_{n=1}^{N_\text{int}}C^n_\alpha\ket{s^n_1,\ldots,s^n_N~;~t^n_1,\ldots,t^n_N}}_{\text{input
together with}~|\lambda|}
\ee

\paragraph{The generic matrix element}
\be\la{eq.genericme}
I_\mathcal{O}(A',\ve{\lambda}',A,\ve{\lambda};P):=
\me{A',\ve{\lambda}'}{\hat{\mathcal{O}}\otimes\hat{\mathcal{I}}}
{\hat{P}(A),\hat{P}(\ve{\lambda})}=\me{A'}{\hat{\mathcal{O}}}
{\hat{P}(A)}\cdot\me{\ve{\lambda}'}{\hat{\mathcal{I}}}
{\hat{P}(\ve{\lambda})}
\ee
with $\hat{P}\in\mathcal{A}$, hence,
\be\la{eq.perma}\hat{P}(A)=T_P^\intercal\,A\,T_P:=A^P\;\;\;.\ee
\be\la{eq.oplist}
\hat{\mathcal{O}}\in\left\lbrace
\mathbb{1}~;~
\ve{p}^\intercal\mathbb{1}_{(3N\times 3N)}\ve{p}~;~
\sumin q_iL_i^z~;~
\sumin q_i(x_i^2+y_i^2+z_i^2)~;~
\sumin q_i\sigma_i^z~;~
\sumijn e^{-\frac{\Lambda^2}{4}(\ve{r}_i-\ve{r}_j)^2}
\right\rbrace\
\ee

\newpage

\begin{landscape}
\paragraph{The matrix elements}

\be\la{tab.mes}
\setlength{\tabcolsep}{4pt}
\renewcommand{\arraystretch}{2.4}
\begin{array}{l|c|c}
\hline
\hat{\mathcal{O}} & 
\me{A'}{\hat{\mathcal{O}}}{\hat{P}(A)} &
\me{\ve{\lambda}'}{\hat{\mathcal{I}}}{\hat{P}(\ve{\lambda})} \\
\hline
% identity
\mathbb{1}:=\mathbb{1}_{\ve{r}}^P\cdot\mathbb{1}_{\ve{s}}^P           &
\left(\frac{(2\pi)^{3N}}
{\det\mathbb{A}_x\det\mathbb{A}_y\det\mathbb{A}_z}\right)^\frac{1}{2} &
\sum\limits_{\alpha,n}^{|\lambda|,N_\text{int}}\lambda_\alpha C^n_\alpha
\overlap{\ve{s}^n;\ve{t}^n}{\hat{P}(\ve{s}^n);\hat{P}(\ve{t}^n)} \\
% E_kin
\frac{1}{2}\ve{p}^\intercal\mathbb{1}_{3N}\ve{p}
=-\frac{\hbar^2}{2}\ve{\nabla}^\intercal\mathbb{1}_{3N}\ve{\nabla}&
\frac{\hbar^2}{2}~\mathbb{1}_{\ve{r}}^P~\prodcart(A_c)_{im}(\mathbb{A}^{-1}_c)_{mn}(A^P_c)_{ni} & \mathbb{1}_{\ve{s}}^P\\
% Lz from background field
\sumin q_iL_i^z=q_i~\left(x_i\partial_{y_i}-y_i\partial_{x_i}\right) &  0 &
\mathbb{Q}^P_{\ve{s}}:=\sumin\sum\limits_{\alpha,n}^{|\lambda|,N_\text{int}}
\underbrace{(\hat{P}[\ve{t}^n])_i}_{=q_{P(i)}} \lambda_\alpha C^n_\alpha
\overlap{\ve{s}^n;\ve{t}^n}{\hat{P}(\ve{s}^n);\hat{P}(\ve{t}^n)}
\\
% osci pot from background field
\sumin q_i(\omega_xx_i^2+\omega_yy_i^2+\omega_zz_i^2)                &
\mathbb{1}_{\ve{r}}^P\prodcart\omega_c\sumin(\mathbb{A}_c^{-1})_{ii} & \mathbb{Q}^P_{\ve{s}} \\
% magnetic moment 
\sumin q_i\sigma_i^z                                                 &
\mathbb{1}_{\ve{r}}^P &
\sumin\sum\limits_{\alpha,n}^{|\lambda|,N_\text{int}}
\underbrace{\left(\hat{P}[\ve{t}^n]\right)_i}_{=q_{P(i)}}~
\underbrace{\left(\hat{P}[\ve{s}^n]\right)_i}_{=s^z_{P(i)}}
\lambda_\alpha C^n_\alpha
\overlap{\ve{s}^n;\ve{t}^n}{\hat{P}(\ve{s}^n);\hat{P}(\ve{t}^n)}  \\
% EFTnoPi LO two-body contact
\sumijn e^{-\frac{\Lambda^2}{4}(\ve{r}_i-\ve{r}_j)^2}                &  \\
% EFTnoPi LO three-body contact
\sumijk e^{-\frac{\Lambda^2}{4}((\ve{r}_i-\ve{r}_j)^2+(\ve{r}_i-\ve{r}_k)^2)}                &  \\
\hline
\end{array}
\ee
with
\begin{align}
\mathbb{A}_x=&A'_x+A^P_x
\end{align}
\end{landscape}
\end{document}