\documentclass[aps,prd,onecolumn
,tightenlines,letterpaper,notitlepage,
%superscriptaddress,
nofootinbib]{revtex4-1}
\usepackage{amssymb,latexsym}
\usepackage{amsmath,amsbsy,bbm}
\usepackage{epsfig,bm,color}
\usepackage{cjhebrew}
\usepackage{nicefrac}
\usepackage{graphicx,comment}
\usepackage{slashed}
%\usepackage{hyperref}
\unitlength=1mm
\DeclareMathOperator{\st}{str}
\DeclareMathOperator{\Erfc}{Erfc}
\DeclareMathOperator{\Erf}{Erf}
\DeclareMathOperator{\Pf}{Pf}
\DeclareMathOperator{\sign}{sign}

\usepackage{dutchcal}

\usepackage{calligra}

\DeclareMathAlphabet{\mathcalligra}{T1}{calligra}{m}{n}
\DeclareFontShape{T1}{calligra}{m}{n}{<->s*[2.2]callig15}{}
\newcommand{\scriptr}{\mathcalligra{r}\,}
\newcommand{\boldscriptr}{\pmb{\mathcalligra{r}}\,}

\begin{document}

\def\a{{\alpha}}
\def\b{{\beta}}
\def\d{{\delta}}
\def\D{{\Delta}}
\def\X{{\Xi}}
\def\e{{\varepsilon}}
\def\g{{\gamma}}
\def\G{{\Gamma}}
\def\k{{\kappa}}
\def\l{{\lambda}}
\def\L{{\Lambda}}
\def\m{{\mu}}
\def\n{{\nu}}
\def\o{{\omega}}
\def\O{{\Omega}}
\def\S{{\Sigma}}
\def\s{{\sigma}}
\def\th{{\theta}}

\def\ol#1{{\overline{#1}}}

\def\Dslash{D\hskip-0.65em /}
\def\Dtslash{\tilde{D} \hskip-0.65em /}

\def\order{{\mathcal O}}

\def\Dt{{\tilde{D}}}
\def\St{{\tilde{\Sigma}}}

\def\eqref#1{{(\ref{#1})}}

\newcommand{\caf}{\text{\cjRL{b}}}
\newcommand{\he}{${}^4$He}
\newcommand{\hes}{${}^3$He}
\newcommand{\tr}{${}^3$H}
\newcommand{\ls}{\ve{L}\cdot\ve{S}}
\newcommand{\eps}{\epsilon}
\newcommand{\as}{a_s}
\newcommand{\at}{a_t}
\newcommand{\ecm}{E_\textrm{\small c.m.}}
\newcommand{\dq}{\mbox{d\hspace{-.55em}$^-$}}
\newcommand{\mpis}{$m_\pi=137~${\small MeV}}
\newcommand{\mpim}{$m_\pi=450~${\small MeV}}
\newcommand{\mpil}{$m_\pi=806~${\small MeV}}
\newcommand{\muh}{\mu_{^3\text{\scriptsize He}}}
\newcommand{\mut}{\mu_{^3\text{\scriptsize H}}}
\newcommand{\mud}{\mu_\text{\scriptsize D}}
\newcommand{\pode}{\beta_{\text{\scriptsize D},\pm1}}
\newcommand{\poh}{\beta_{^3\text{\scriptsize He}}}
\newcommand{\pot}{\beta_{^3\text{\scriptsize H}}}
\newcommand{\com}[1]{{\scriptsize \sffamily \bfseries \color{red}{#1}}}
\newcommand{\eg}{\textit{e.g.}\;}
\newcommand{\ie}{\textit{i.e.}\;}
\newcommand{\cf}{\textit{c.f.}\;}
\newcommand{\be}{\begin{equation}}
\newcommand{\ee}{\end{equation}}
\newcommand{\la}{\label}
\newcommand{\ber}{\begin{eqnarray}}
\newcommand{\eer}{\end{eqnarray}}
\newcommand{\nn}{\nonumber}
\newcommand{\half}{\frac{1}{2}}
\newcommand{\thalf}{\nicefrac[]{3}{2}}
\newcommand{\bs}[1]{\ensuremath{\boldsymbol{#1}}}
\newcommand{\bea}{\begin{eqnarray}}
\newcommand{\eea}{\end{eqnarray}}
\newcommand{\beq}{\begin{align}}
\newcommand{\eeq}{\end{align}}
\newcommand{\bk}{\bs k}
\newcommand{\bt}{B_{^{3}\text{H}}}
\newcommand{\bh}{B_{^{3}\text{He}}}
\newcommand{\bd}{B_\text{D}}
\newcommand{\ba}{B_\alpha}
\newcommand{\rgm}{$\mathbb{R}$GM}
\newcommand{\ev}[1] {|\bra #1  \ket |^2}
\newcommand{\lam}[1]{$\Lambda=#1~$fm$^{-1}$}
\newcommand{\parg}[1] {\paragraph*{-\,\textit{#1}\,-}}
\newcommand{\nopi}{\pi\hspace{-6pt}/}
\newcommand{\ve}[1]{\ensuremath{\boldsymbol{#1}}}
\newcommand{\xvec}{\bs{x}}
\newcommand{\rvec}{\bs{r}}
\newcommand{\sgve}{\ensuremath{\boldsymbol{\sigma}}}
\newcommand{\tave}{\ensuremath{\boldsymbol{\tau}}}
\newcommand{\na}{\nabla}
\newcommand{\bra}{\langle}
\newcommand{\ket}{\rangle}
\newcommand{\tx}{\tilde{x}}
\newcommand{\eftnopi}{\mbox{EFT($\slashed{\pi}$)}}

\newcommand{\cmment}[2]{\paragraph*{Ecce: #1}\texttt{\textcolor{blue}{#2}}}



\author{$\mathcal{M}$.~$\mathcal{Elyahu}$}
\author{$\mathcal{N}$.~$\mathcal{Barnea}$}
\author{$\mathcal{J}$.~$\mathcal{Kirscher}$}
%\email[]{$\texttt{bctiburz@chinamail.cn}$}
%\affiliation{%
%$\mathcal{CCNY}$ (Consolidated Chinese National Yeshiva)}

\title{
Strong magnetic fields and contact interactions in few-fermion systems
} 

\begin{abstract}
The effect of a magnetic background on few-nucleon systems is analyzed. In particular, the spectra of pure proton systems
with up to 5 constituents are calculated as functions of the strength of the magnetic field. From these spectra, we
conjecture the critical field strength which stabilizes the general $N$-proton system. For $N\gtrsim 10$, the magnetic field in
the vicinity of a typical neutron star is found sufficient to form a bound structure.

Furthermore, the dependence of the tri-nucleon binding energy splitting is calculated in the presence of the magnetic
field. Combining this result with the effect of the background on phase space available to the $\beta$-decay products
of the triton, we predict its $\beta$-stabilization -- and thereby the instability of 3-helium -- at field strengths of $B=\mathcal{O}(10^{10}\text{T})$.

For particle-stable systems in vacuum, field strengths are calculated which preclude this stability. Qualitatively, this result is found
consistent with LQCD simulations, employing an unphysically large pion mass.

All results are based on numerical solutions of the Schr\"odinger equation with a nuclear-interaction as approximated by the leading order
of the \eftnopi. 
\end{abstract}

\pacs{}

\maketitle

\section{Summary of $\langle\text{\cjRL{ywhns}}\vert\text{\cjRL{mwty}}\rangle$}

\begin{enumerate}
\item We realized the need for a complete SVM basis $\phi_{AsB}(\underbrace{x_1,\ldots,x_N}_{:=\ve{x}})=e^{-\frac{1}{2}\ve{x}^TA\ve{x}+\frac{1}{2}(\ve{x}-\ve{s})^TB(\ve{x}-\ve{s})}$

\end{enumerate}

\section{Benchmarking}

The Hamiltonian which describes the nuclear few-body problem in leading-order
\eftnopi, in a magnetic background field which interacts via minimal, one-nucleon
coupling, only, is given by Eq.\eqref{eq.hamiltonian}. The correct implementation of
this operator is verified as follows.

\begin{description}
\item{Nuclear two- and three-body interaction} We reproduce the deuteron and triton binding energies for sets of
pre-calibrated low-energy constants in the absence of the magnetic field. Furthermore, cutoff-independent postdictions
for the binding energy of the $\alpha$ particle are compared with those of RGM, MC, and EIHH techniques.

Finally, we detail the evaluation of the matrix elements of a generic pair/triplet of interacting
particles via Gaussian-regulated two-/three-body contact interactions between one representative of
the SVM basis both, analytically and numerically. This step serves educational and benchmarking purposes.
\be\la{eq.svmME}
\bra~A'B's'~\vert~\hat{V}_{2(3)}~\vert~A,B,s~\ket=\ldots\stackrel{N=4}{=}
\ee

\item{Magnetic $xy$ trap} For non-interacting particles, we obtain an energy spectrum whose level splitting
matches the Landau spectrum to be expected if all, or a subset of the systems carries charges. For the two-body system,
we show explicitly that for $\lim\limits_{q_1/q_2\to 0}$ the single-particle Landau spectrum is attained.

\item{Angular $z$ momentum (orbit)}


\item{Angular $z$ momentum (spin)}

\end{description}

\section{Theoretical description of strongly interacting fermions in a magnetic
background field}

In order to describe the effect of a static magnetic field, which is not necessarily
perturbatively small, on the low-energy spectra of small nuclei, the 
effective-field-theory Hamiltonian of the latter is augmented by, first, minimally coupling
the derivatives, and second, by the magnetic-moment 1-body interaction (a remnant of the
non-relativistic reduction of the Dirac equation) and the coupling of the $\ve{B}$ field
to the four-nucleon leading-order momentum-independent vertex.

As of now, we do consider minimal coupling, only. In addition, we assume the nuclear interaction
to be comprised of a two-body contact interaction which yields a shallow spin-$1$ deuteron. The
interaction in the spin-$0$ (iso-spin-$1$) channel is also attractive but not strong enough to
sustain a bound di-neutron/proton. In order to stabilize the three-nucleon system in the 
spin-$\frac{1}{2}$ (triton) channel, a
three-nucleon contact term must be included. Using the so-called symmetric gauge,
$\ve{A}_i=\frac{B_0}{2}(-y_i,x_i,0)$,
and regulating the contact interactions with Gaussian functions,
the dynamics of an $A$-particle system are dictated by the following Hamilton operator:

\begin{align}\la{eq.hamiltonian}
\hat{H}=-\frac{\hbar^2}{2m}&\sum_i\Bigg\lbrace
\ve{\nabla}_i^2
+i\left(\frac{\hbar^2}{2m}\right)\left(\frac{q_iB_0}{\hbar}\right)
\left(x_i\partial_{y_i}-y_i\partial_{x_i}\right)
+\left(\frac{\hbar^2}{2m}\right)\left(\frac{q_iB_0}{\hbar}\right)^2\frac{1}{4}
\left(x_i^2+y_i^2\right)
-g_i\left(\frac{\hbar^2}{2m}\right)\left(\frac{q_iB_0}{\hbar}\right)\sigma_{z_i}
\Bigg\rbrace\\
&+~\sum_{i<j}\left[C_1^\Lambda~\hat{P}({}^1S_0) + C_2^\Lambda~\hat{P}({}^3S_1)\right]~e^{-\frac{\Lambda^2}{4}(\ve{r}_i-\ve{r}_j)^2}\nonumber\\
&+\sum_{i<j<j\atop\text{cyc.}}D_0^\Lambda~\hat{P}(S=\nicefrac{1}{2})
~e^{-\frac{\Lambda^2}{4}\left((\ve{r}_i-\ve{r}_j)^2+(\ve{r}_i-\ve{r}_k)^2\right)}
\end{align}

\subsection{Notable features}

\begin{enumerate}
\item Total Spin in not conserved, which means that if we consider a 2-nucleon
system, it resides in a superposition of spin-1 and spin-0 states.
\item Bound states might not have negative eigenvalues because the non-interacting
theory still contains the effect of the magnetic field on the charged particles
and shifts their ground state by, at least, the energy in the lowest Landau level.
\item
\end{enumerate}

\section{Two-nucleon systems}

First, we analyse the predictions in Ref.~\cite{PhysRevLett.116.112301}, \ie, what is
the dependence of the lowest eigenvalues of the 2-proton system on the strength of the
magnetic field?

\section{Three Nucleons}

How does the critical field strength at which the 3-proton system becomes bound compare
with the corresponding strength which is necessary to bind two protons?

\section{Four Nucleons}

Same critical-strength ratio? Now we extrapolate and predict that no more than 
$\vert\ve{B}\vert\sim 4~$T is needed to bind an $A$-proton cluster.

\appendix*
\section{Single-particle stochastic-variational method}\la{app.svm}
We expand an $N$-body wave function in one Cartesian dimension $x$ in a correlated Gaussian basis with
basis vectors

\be\la{eq.1dbasis}
\phi_A(\underbrace{x_1,\ldots,x_N}_{:=\ve{x}})=e^{\ve{x}^TA\ve{x}}
\ee

parametrized by a symmetric $N\times N$ matrix $A=A^T$.

\bibliography{refs}

\end{document}